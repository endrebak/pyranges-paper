\documentclass[10pt,letterpaper]{article}
\usepackage[top=0.85in,left=2.75in,footskip=0.75in,marginparwidth=2in]{geometry}
\usepackage{listings}

% use Unicode characters - try changing the option if you run into troubles with special characters (e.g. umlauts)

\usepackage[utf8]{inputenc}


% clean citations
\usepackage{cite}

% hyperref makes references clicky. use \url{www.example.com} or \href{www.example.com}{description} to add a clicky url
\usepackage{nameref,hyperref}

% line numbers
\usepackage[right]{lineno}

% improves typesetting in LaTeX
\usepackage{microtype}
\DisableLigatures[f]{encoding = *, family = * }

% text layout - change as needed
\raggedright
\setlength{\parindent}{0.5cm}
\textwidth 5.25in
\textheight 8.75in

% use adjustwidth environment to exceed text width (see examples in text)
\usepackage{changepage}

% adjust caption style

% remove brackets from references
\makeatletter
\renewcommand{\@biblabel}[1]{\quad#1.}
\makeatother

% headrule, footrule and page numbers
\usepackage{lastpage,fancyhdr,graphicx}
\usepackage{epstopdf}
\pagestyle{myheadings}
\pagestyle{fancy}
\fancyhf{}
\rfoot{\thepage/\pageref{LastPage}}
\renewcommand{\footrule}{\hrule height 2pt \vspace{2mm}}
\fancyheadoffset[L]{2.25in}
\fancyfootoffset[L]{2.25in}

% use \textcolor{color}{text} for colored text (e.g. highlight to-do areas)
\usepackage{color}

% define custom colors (this one is for figure captions)
\definecolor{Gray}{gray}{.25}

% this is required to include graphics
\usepackage{graphicx}

% use if you want to put caption to the side of the figure - see example in text
\usepackage{sidecap}

% use for have text wrap around figures
\usepackage{wrapfig}
\usepackage[pscoord]{eso-pic}
\usepackage[fulladjust]{marginnote}
\reversemarginpar

% document begins here
\begin{document}
\vspace*{0.35in}

% title goes here:
\begin{flushleft}
{\Large
  \textbf\newline{pyranges: an efficient and flexible library for genomics in Python}
  % and arithmetic run length-encoding
}
\newline
% authors go here:
\\
Endre Bakken Stovner\textsuperscript{1},
Pål Sætrom\textsuperscript{1, 2},
\\
\bf{1} Department of
  Computer Science, Norwegian University
  of Science and Technology, Trondheim, 7013, Norway
\\
\bf{2} Department of Clinical and Molecular Medicine, Norwegian
  University of Science and Technology, Trondheim, 7013, Norway
\\
\bf{3} Bioinformatics Core Facility, Norwegian University of Science and
Technology, Trondheim, 7013, Norway
\\
\bf{4} K.G. Jebsen Center for Genetic Epidemiology, Department of Public Health
and Nursing, Norwegian University of Science and Technology, Trondheim, Norway
\\
\bigskip
* endrebak85@gmail.com

\end{flushleft}

% A Correspondence (formerly Letters to the Editor) is a flexible format that may include anything of interest to the journal's readers, from policy debates to announcements to 'matters arising' from research papers. A Correspondence may describe primary research data, but only in summary form; this format is not intended for full presentation of data. Correspondence should never be more than one printed page, and usually much less. The number of references should not exceed 10 for either the Correspondence or its Reply, and article titles are omitted from the reference list. Titles for correspondence are supplied by the editors.

% In cases where a correspondence is critical of a previous research paper, the authors are normally given the option of publishing a brief reply. Criticism of opinions or other secondary matter does not involve an automatic right of reply.

% Refutations are always peer reviewed. Other types of Correspondence may be peer reviewed at the editors' discretion.

\subsection*{To the Editor:} Complex genomic analyses are carried out by
combining simple operations like intersection, overlap and nearest. Still,
genomic analyses are often time-consuming and error-prone to write for the
individual researcher as there are many details to get right. And the memory and
speed performance is often low, as performance requires a deep understanding of
lower-level aspects of programming. Low performance is annoying for a one-off
script, but inefficient and wasteful of resources for published software that is
run many times. And as genomic analyses are trivially parallelizeable per
chromosome, genomic software should take advantage of this fact to produce a
speedup equal to the time it takes to run the code on the longest chromosome.

To this end, we have written the library PyRanges. PyRanges contains an
efficient datastructure to represent genomic intervals and their associated
metadata. PyRanges also contains efficient methods to operate on single
intervals and compare pairs of intervals. Furthermore, PyRanges contains an
extremely efficient run-length encoding library for arithmetic computations on
the coverage (or any nucleotide-associated score) associated with one or more
read libraries.

PyRanges is not the only library that provides a datastructure with many methods
for comparing intervals. In R, the library GenomicRanges
\cite{10.1371/journal.pcbi.1003118} provides the foundation for virtually all
Genomics packages in the R BioConductor \cite{Gentleman2004} project. However,
this library is slow for large datasets and is not automatically parallelized
and thus do not meet the ideal requirements for genomics libraries. Furthermore,
as Python is perhaps the most used programming language in the world and
immensely popular in bioinformatics, a GenomicRanges-library in Python would
likely be a boon to the bioinformatics community even without the added
efficiency.

Benchmarks show that PyRanges is much faster (how?) and typically requires 50
percent less memory than BioConductor GenomicRanges.

To leverage the enormous Python ecosystem for scientific computing the PyRanges
data are stored in Pandas dataframes, which means that one can seamlessly use
Pythons wealth of data science libraries on the data contained in PyRanges
objects. For the operations that require range-queries or overlaps, the Nested
Containment List \cite{doi:10.1093/bioinformatics/btl647} is used. PyRanges is
by default single-threaded, but by adding one line of initialization, multiple
cores and even clusters can be used.

To ensure correctness PyRanges has been extensively tested using property-based
testing. Property-based testing is a method to generate data matching some
specification and doing tests on these. The data generated is random genomic
data, while the tests ensure that PyRanges gives the same result as the
equivalent functionality in the mature and proven libraries
bedtools\cite{doi:10.1093/bioinformatics/btq033} and BioConductor GenomicRanges.
This means that instead of writing a few tests by hand, the computer is set to
disprove that your code works by finding quirky edge-cases you would not have
thought to look for.



\bibliography{library}

\bibliographystyle{abbrv}

\end{document}

% easy to add new functionality
% points: automatically choose correct representation of data: 2x speedup, better memory use