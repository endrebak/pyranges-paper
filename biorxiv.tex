\documentclass[10pt,letterpaper]{article}
\usepackage[top=0.85in,left=2.75in,footskip=0.75in,marginparwidth=2in]{geometry}

% use Unicode characters - try changing the option if you run into troubles with special characters (e.g. umlauts)

\usepackage[utf8]{inputenc}


% clean citations
\usepackage{cite}

% hyperref makes references clicky. use \url{www.example.com} or \href{www.example.com}{description} to add a clicky url
\usepackage{nameref,hyperref}

% line numbers
\usepackage[right]{lineno}

% improves typesetting in LaTeX
\usepackage{microtype}
\DisableLigatures[f]{encoding = *, family = * }

% text layout - change as needed
\raggedright
\setlength{\parindent}{0.5cm}
\textwidth 5.25in
\textheight 8.75in

% use adjustwidth environment to exceed text width (see examples in text)
\usepackage{changepage}

% adjust caption style

% remove brackets from references
\makeatletter
\renewcommand{\@biblabel}[1]{\quad#1.}
\makeatother

% headrule, footrule and page numbers
\usepackage{lastpage,fancyhdr,graphicx}
\usepackage{epstopdf}
\pagestyle{myheadings}
\pagestyle{fancy}
\fancyhf{}
\rfoot{\thepage/\pageref{LastPage}}
\renewcommand{\footrule}{\hrule height 2pt \vspace{2mm}}
\fancyheadoffset[L]{2.25in}
\fancyfootoffset[L]{2.25in}

% use \textcolor{color}{text} for colored text (e.g. highlight to-do areas)
\usepackage{color}

% define custom colors (this one is for figure captions)
\definecolor{Gray}{gray}{.25}

% this is required to include graphics
\usepackage{graphicx}

% use if you want to put caption to the side of the figure - see example in text
\usepackage{sidecap}

% use for have text wrap around figures
\usepackage{wrapfig}
\usepackage[pscoord]{eso-pic}
\usepackage[fulladjust]{marginnote}
\reversemarginpar

% document begins here
\begin{document}
\vspace*{0.35in}

% title goes here:
\begin{flushleft}
{\Large
  \textbf\newline{pyranges: efficient comparisons of genomic intervals in Python}
  % and arithmetic run length-encoding
}
\newline
% authors go here:
\\
Endre Bakken Stovner\textsuperscript{1},
Pål Sætrom\textsuperscript{1, 2},
\\
\bf{1} Department of
  Computer Science, Norwegian University
  of Science and Technology, Trondheim, 7013, Norway
\\
\bf{2} Department of Clinical and Molecular Medicine, Norwegian
  University of Science and Technology, Trondheim, 7013, Norway
\\
\bigskip
* endrebak85@gmail.com

\end{flushleft}

\section*{Abstract}

\textbf{Summary:} PyRanges is a high-performance datastructure for representing
and manipulating genomic ranges and their associated data in Python. It is
directly compatible with Python's wealth of high-performance data science
libraries and is therefore well suited for genomic analyses.

\textbf{Availability and Implementation:} PyRanges is available open-source under
the MIT license at https://github.com/endrebak/pyranges

\textbf{Contact:} endrebak85@gmail.com

\section*{Introduction}

Comparing sets of intervals is a fundamental task in genomics, and a few basic
operations allow for answering complex questions. Several very popular command
line tools for this purpose exist, like bedtools
\cite{doi:10.1093/bioinformatics/btq033} and bedops
\cite{doi:10.1093/bioinformatics/bts277}. There is also a wrapper for bedtools
which allows easy use from Python \cite{doi:10.1093/bioinformatics/btr539}.

Since comparing sets of intervals is such a fundamental task, the datastructure
GenomicRanges for efficiently representing and operating on genomic intervals
was invented. This datastructure removes the need for library authors to come up
with their own custom, slow and hard to maintain solutions and allows them to
immediately begin solving their problem of interest. Indeed, in R, the
foundational GenomicRanges library \cite{10.1371/journal.pcbi.1003118} is used
for feature comparisons, and it is a cornerstone of genomics packages in the R
BioConductor \cite{Gentleman2004} project. The new Julia language
\cite{doi:10.1137/141000671} also has a package for GenomicRanges
\cite{Haverty2017}.

Python is the fourth largest programming language in the world, the largest
scripting language, and it is widely used in bioinformatics, yet Python lacks a
GenomicRanges implementation. The PyRanges library remedies this.

\section*{Library}

PyRanges is a thin wrapper around genomic data contained in Pandas
\cite{mckinney-proc-scipy-2010} dataframes. This allows one to directly use
Python's wealth of libraries for scientific computing and plotting such as
SciPy, NumPy, scikit-learn, matplotlib and seaborn \cite{scipy},
\cite{oliphant-2006-guide}, \cite{scikit-learn}, \cite{Hunter:2007},
\cite{michael_waskom_2017_883859} on the data in PyRanges objects. However, the
PyRanges object is extended with many methods for subsetting and modifying it
and for doing comparisons and queries with pairs of PyRanges. PyRanges also
contains coverage methods, which allows one to represent and do arithmetic
computations on the coverage (or some other numeric score associated with each
nucleotide) in an extremely efficient manner. These run length encodings are
stored in an object called PyRles. PyRanges includes readers for common
file formats in bioinformatics such as bed, bam and gtf.

% \begin{table}[!ht]
% \begin{adjustwidth}{-1.5in}{0in}
% \centering
% \caption{{\bf Functionality.} Methods available on PyRanges objects.}
% \begin{tabular}{|l|l|l|l|l|l|l|}
% \hline
%   {\bf Method} & {\bf Purpose} \\ \hline
%   Subset & Use Python's getitem operator to efficiently subset PyRanges \\ \hline
%   Intersection & For each feature in self, keep those that overlap with intervals in other \footnotemark \\ \hline
%   Set Intersection & Treat the intervals in each PyRanges as a set and find the intersection \footnotemark \\ \hline
%   Overlap & Keep each feature in self that overlaps with at least one in other \\ \hline
%   Set Union & Treat the intervals in each PyRanges as a set and find the union \\ \hline
%   Subtract & For each feature in self, keep those parts that do not overlap with any feature in other \\ \hline
%   Set Subtract & Treat the intervals in each PyRanges as a set and subtract other from self \\ \hline
%   Nearest & For each feature in self, find the nearest in other \\ \hline
%   Cluster & Merge the overlapping features in a PyRanges object \\ \hline
%   Coverage & Create a dictionary of run length encodings from the PyRanges object \\ \hline
% \end{tabular}
% \label{tab1}
% \end{adjustwidth}
% \end{table}

% \footnotetext{Like bedtools intersect}
% \footnotetext{Like BioConductor GenomicRanges intersect}

\subsection*{Implementation}

PyRanges uses the nested containment list, originally implemented in the PyGr
library \cite{doi:10.1093/bioinformatics/btl647} for range based lookups and
queries. The pyranges library is written in pure Python for rapid development
and testing, but it depends on several C/Cython packages written to make
PyRanges very efficient.

\subsection*{Advantages}

In contrast to PyRanges, which stores the data in memory, PyBedtools uses disk
space for storing temporary results, which requires additional resource
management by users and can lead to resource leaks; e.g. if the process calling
PyBedtools is killed.

In contrast to bioconductor GenomicRanges, operations on PyRanges returns new
PyRanges-objects, not indices.

\subsection*{Timings}



\section*{Discussion}

PyRanges is a library that works well with the existing Python data science
libraries yet extends them with high-performance functionality for genomic
analyses. As such we believe it will be useful for bioinformatics scripts and
libraries written in Python.

\bibliography{library}

\bibliographystyle{abbrv}

\end{document}
